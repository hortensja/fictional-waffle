% !TEX TS-program = pdflatex
% !TEX encoding = UTF-8 Unicode

% This is a simple template for a LaTeX document using the "article" class.
% See "book", "report", "letter" for other types of document.

\documentclass[10pt]{article} % use larger type; default would be 10pt
\linespread{1.3}
\usepackage{polski}
\usepackage[utf8]{inputenc} % set input encoding (not needed with XeLaTeX)
\usepackage[letterpaper, landscape, margin=2.5cm]{geometry}
%\usepackage{helvet}
\usepackage{indentfirst}
%\renewcommand{\familydefault}{\sfdefault}
%%% Examples of Article customizations
% These packages are optional, depending whether you want the features they provide.
% See the LaTeX Companion or other references for full information.

%%% PAGE DIMENSIONS
\usepackage{geometry} % to change the page dimensions
\geometry{a4paper} % or letterpaper (US) or a5paper or....
% \geometry{margin=2in} % for example, change the margins to 2 inches all round
% \geometry{landscape} % set up the page for landscape
%   read geometry.pdf for detailed page layout information

\usepackage{graphicx} % support the \includegraphics command and options

% \usepackage[parfill]{parskip} % Activate to begin paragraphs with an empty line rather than an indent

%%% PACKAGES
\usepackage{booktabs} % for much better looking tables
\usepackage{array} % for better arrays (eg matrices) in maths
\usepackage{paralist} % very flexible & customisable lists (eg. enumerate/itemize, etc.)
\usepackage{verbatim} % adds environment for commenting out blocks of text & for better verbatim
\usepackage{subfig} % make it possible to include more than one captioned figure/table in a single float
\usepackage{multicol}
\usepackage{amsmath}
\usepackage{bm}
\usepackage{amsthm}
\usepackage{ bbold }
%\usepackage{braket}
% These packages are all incorporated in the memoir class to one degree or another...

%%% HEADERS & FOOTERS
\usepackage{fancyhdr} % This should be set AFTER setting up the page geometry
\pagestyle{fancy} % options: empty , plain , fancy
\renewcommand{\headrulewidth}{0pt} % customise the layout...
\lhead{}\chead{}\rhead{}
\lfoot{}\cfoot{\thepage}\rfoot{}
	
%%% SECTION TITLE APPEARANCE
\usepackage{sectsty}
\allsectionsfont{\sffamily\mdseries\upshape} % (See the fntguide.pdf for font help)
% (This matches ConTeXt defaults)

%%% ToC (table of contents) APPEARANCE
\usepackage[nottoc,notlof,notlot]{tocbibind} % Put the bibliography in the ToC
\usepackage[titles,subfigure]{tocloft} % Alter the style of the Table of Contents
\renewcommand{\cftsecfont}{\rmfamily\mdseries\upshape}
\renewcommand{\cftsecpagefont}{\rmfamily\mdseries\upshape} % No bold!

%%% END Article customizations

%%% Math ops %%%
\DeclareMathOperator{\Trs}{Tr}

%%% END

%%% Extra commands %%%
\newcommand{\Mats}[1]{\mathcal{L}(#1)}
\newcommand{\Hx}[1]{\mathcal{H}^{#1}}
\newcommand{\LHx}[1]{\Mats{\Hx{#1}}}
\newcommand{\HAi}{\Hx{A_1}}
\newcommand{\LHAi}{\Mats{\HAi}}
\newcommand{\MXi}[3]{\mathcal{M}^{#1}_{#2}(#3)}
\newcommand{\MXin}[2]{\mathcal{M}^{#1}_{#2}}
\newcommand{\MAin}[0]{\MXin{A}{i}}
\newcommand{\MAi}[1]{\MXi{A}{i}{#1}}
\newcommand{\MAir}{\MAi{\rho}}
\newcommand{\Idx}[1]{\mathbb{1}^{#1}}
\newcommand{\Tr}[1]{\Trs(#1)}
\newcommand{\Pro}[1]{\Pr(#1)}
\newcommand{\Prt}[2]{\Pr(#1, #2)}
\newcommand{\Ket}[1]{|#1\rangle}
\newcommand{\Bra}[1]{\langle#1|}
\newcommand{\CP}{\textit{completely positive}}
\newcommand{\TP}{\textit{trace preserving}}
\newcommand{\CPTP}{\textit{completely posite trace preserving}}
\newcommand{\WAll}{W^{A_1A_2B_1B_2}}
\newcommand{\MA}{M^{A_1A_2}}
\newcommand{\MB}{M^{B_1B_2}}
\newcommand{\mai}[1]{\MA_{#1}}
\newcommand{\mbi}[1]{\MB_{#1}}
%%% END

%%% The "real" document content comes below...

\title{Przyczynowe więzy na strukturę korelacji w formalizmie kwantowym}
\author{Piotr Krasuń}
%\date{} % Activate to display a given date or no date (if empty),
         % otherwise the current date is printed 
\setlength{\parindent}{1.25cm}
\begin{document}
\maketitle
\tableofcontents
\newpage
\listoffigures
\listoftables
\newpage
%\begin{multicols*}{2}
\section{Wstęp}
Tekst wstępu
\section{Macierz Procesu}
Jednym z podejść do eksploracji korelacji nie zachowujących przyczynowego porządku jest rozwinięty w \cite{process_matrix} formalizm macierzy procesu.
Ewidentną zaletą tego podejścia jest zgodność z mechanika kwantową na poziomie lokalnych eksperymentów. Jest to nijakie rozszerzenie i enkapsulacja idei POVM i reguły Borna. Podejście te porzuca założenie globalnej struktury czasoprzestrzeni. W celu zachowania zgodności z mechaniką kwantową na poziomie lokalnym opiera się na następujących założeniu, że operacje wykonywane przez poszczególną stronę są opisywane przez mechanikę kwantową w standardowym przyczynowym sformułowaniu, które można opisywać przy pomocy zbioru \textit{quantum instruments} \cite{quantum_instrument} z wejściową przestrzenią Hilberta $\mathcal{H}^{A_1}$ i przestrzenią wyjściową  $\mathcal{H}^{A_2}$. Najogólniej można je realizować przy pomocy zadziałania unitarną transformacją na system wejściowy i \textit{ancilla}, następnie wykonanie rzutującego pomiaru na części systemu pozostawiając pozostała część systemu jako wyjście. Alicja wykorzystując dany instrument otrzymuje jeden z możliwych wyników $x_i$, który indukuję transformację $\mathcal{M}^A_i$ z wejścia na wyjście. Transformacja ta odpowiada \CP~(CP) \TP~mapie\footnote
{
Liniową mapę $\phi$ nazywamy CP, gdy $a \geq 0\implies \phi(a) \geq 0$ oraz $\forall_{k \in \mathcal{N}}~\mathcal{I}_k \otimes \phi$ również jest CP
}
\begin{equation}
\mathcal{M}^A_i : \mathcal{L}(\mathcal{H}^{A_1}) \mapsto \mathcal{L}(\mathcal{H}^{A_2})
\end{equation}
gdzie $\LHx{X}$ jest przestrzenią macierzy na $\Hx{X}$, której wymiar to $d_X$. Jej działanie na macierz gęstości $\rho$ opisuje następująca formuła
\begin{equation}
\label{yolo}
\MAi{\rho} = \sum^m_{j=1} E_{ij} ^\dag \rho E_{ij}
\end{equation}
gdzie macierze $E_{ij}$ spełniają następujące własności 
\begin{gather}
\sum^m_{i=0} E_{ij}^\dag E_{ij} \leq \Idx{A_1} \\
\label{eq:id_proj} 
\sum^n_{i=0} \sum^m_{j=0} E_{ij}^\dag E_{ij} = \Idx{A_1}
\end{gather}
Prawdopodobieństwo zaobserwowania wyniku realizowanego przez mapę $\MAin$ to
\begin{equation}
\Pro{\MAin} = \Tr{\MAir}
\end{equation}
Widzimy od razu, że równanie \eqref{eq:id_proj} narzuca, by możliwość zaobserwowania dowolnego wyniku była równa 1.
W przypadku, gdy mamy do czynienia z więcej niż jedną stroną \textit{procesem} będziemy nazywać listę $\Pr(\MXin{A}{i}, \MXin{B}{j}, \dots)$ dla wszystkich możliwych lokalnych wyników. Dalej będę opisywał wyłącznie przypadek dwustronny, jednakże rozszerzenie formalizmu na przypadek wielostronny jest trywialny. Wygodnym sposobem przedstawiania map $\MAin$ jest izomorfizm Choi-Jamiołkowsky (CJ) \cite{cj_iso1, cj_iso2}. Macierz CJ $M^{A_1A_2}_i \in \Mats{\Hx{A_1} \otimes \Hx{A_2}} \geq 0$ jest zdefiniowana jako
\begin{gather}
\label{eq:cj_iso}
M^{A_1A_2}_i := [\mathcal{I} \otimes \MAi{ \Ket{\phi^+} \Bra{\phi^+}}]^T, \\
\Ket{\phi^+} = \sum^{d_{A_1}}_{i=1} \Ket{ii},
\end{gather}
gdzie $\{\Ket{j}\}^{d_{A_1}}$ tworzy ortonormalna bazę w $\HAi$. Korzystając z tego przejścia można zapisać prawdopodobieństwo dwóch rezultatów, jako 
\begin{equation}
\label{eq:cj_prob}
\Prt{\MAin}{\MXin{B}{j}} = \Tr{\WAll(M^{A_1A_2}_i \otimes M^{B_1B_2}_j)}.
\end{equation}
Macierz $W$ w $\Mats{\Hx{A_1} \otimes \Hx{A_2} \otimes \Hx{B_1} \otimes \Hx{B_2}}$ nazywa się \textit{process matrix} (macierzą procesu).
W celu generowania prawidłowego prawdopodobieństwa narzuca się dodatkowe warunki na $W$
\begin{gather}
\label{eq:non_neg}
\WAll \geq 0. \\
\label{eq:sums_to_id1}
\Trs
\left[
\WAll
\left(
M^{A_1A_2} \otimes M^{B_1B_2}
\right)
\right]=1.\\
\label{eq:sums_to_id2}
\forall M^{A_1A_2}, M^{B_1B_2} \geq 0, \Trs_{A_2} \MA = \mathbb{1}^{A_1}, \Trs_{B_2} \MB = \mathbb{1}^{B_1},
\end{gather}
gdzie $\MA = \sum_i \mai{i}$. Warunek \eqref{eq:non_neg} zapewnia, że prawdopodobieństwa nie będą ujemne, a \eqref{eq:sums_to_id1} i  \eqref{eq:sums_to_id2} pewność zaobserwowania dowolnej pary map. 
%tutej moze macierz gestosci
\begin{figure}[t]
\centering
\label{fig:configs}
\begin{tabular}{|c|c|c|c|}
\hline
Przyczynowy porządek & Stany & Kanały & Kanały z pamięcią \\
$A \npreceq B $ &
\hline
$B \npreceq B $ &
\includegraphics{
\hline
\end{tabular}
\end{figure}
%\end{multicols*}
\newpage
\bibliographystyle{plain}
\bibliography{bibliografia}

\end{document}
