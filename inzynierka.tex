% !TEX TS-program = pdflatex
% !TEX encoding = UTF-8 Unicode

% This is a simple template for a LaTeX document using the "article" class.
% See "book", "report", "letter" for other types of document.

\documentclass[11pt]{article} % use larger type; default would be 10pt
\usepackage{polski}
\usepackage[utf8]{inputenc} % set input encoding (not needed with XeLaTeX)

%%% Examples of Article customizations
% These packages are optional, depending whether you want the features they provide.
% See the LaTeX Companion or other references for full information.

%%% PAGE DIMENSIONS
\usepackage{geometry} % to change the page dimensions
\geometry{a4paper} % or letterpaper (US) or a5paper or....
% \geometry{margin=2in} % for example, change the margins to 2 inches all round
% \geometry{landscape} % set up the page for landscape
%   read geometry.pdf for detailed page layout information

\usepackage{graphicx} % support the \includegraphics command and options

% \usepackage[parfill]{parskip} % Activate to begin paragraphs with an empty line rather than an indent

%%% PACKAGES
\usepackage{booktabs} % for much better looking tables
\usepackage{array} % for better arrays (eg matrices) in maths
\usepackage{paralist} % very flexible & customisable lists (eg. enumerate/itemize, etc.)
\usepackage{verbatim} % adds environment for commenting out blocks of text & for better verbatim
\usepackage{subfig} % make it possible to include more than one captioned figure/table in a single float
\usepackage{multicol}
% These packages are all incorporated in the memoir class to one degree or another...

%%% HEADERS & FOOTERS
\usepackage{fancyhdr} % This should be set AFTER setting up the page geometry
\pagestyle{fancy} % options: empty , plain , fancy
\renewcommand{\headrulewidth}{0pt} % customise the layout...
\lhead{}\chead{}\rhead{}
\lfoot{}\cfoot{\thepage}\rfoot{}
	
%%% SECTION TITLE APPEARANCE
\usepackage{sectsty}
\allsectionsfont{\sffamily\mdseries\upshape} % (See the fntguide.pdf for font help)
% (This matches ConTeXt defaults)

%%% ToC (table of contents) APPEARANCE
\usepackage[nottoc,notlof,notlot]{tocbibind} % Put the bibliography in the ToC
\usepackage[titles,subfigure]{tocloft} % Alter the style of the Table of Contents
\renewcommand{\cftsecfont}{\rmfamily\mdseries\upshape}
\renewcommand{\cftsecpagefont}{\rmfamily\mdseries\upshape} % No bold!

%%% END Article customizations

%%% Extra commands %%%
\newcommand{\Mats}[1]{\mathcal{L}(#1)}
\newcommand{\Hx}[1]{\mathcal{H}^{#1}}
\newcommand{\LHx}[1]{\Mats{\Hx{#1}}}
\newcommand{\HAi}{\Hx{A_1}}
\newcommand{\LHAi}{\Mats{\HAi}}
\newcommand{\MXi}[3]{\mathcal{M}^{#1}_{#2}(#3)}
\newcommand{\MAi}[1]{\MXi{A}{i}{#1}}
%%% END

%%% The "real" document content comes below...

\title{Przyczynowe więzy na strukturę korelacji w formalizmie kwantowym}
\author{Piotr Krasuń}
%\date{} % Activate to display a given date or no date (if empty),
         % otherwise the current date is printed 

\begin{document}
\maketitle
\tableofcontents
\newpage
\listoffigures
\listoftables
\newpage
\begin{multicols*}{2}
\section{Wstęp}

\section{Macierz Procesu}
Jednym z podejść do eksploracji korelacji nie zachowujących przyczynowego porządku jest rozwinięty w \cite{process_matrix} formalizm macierzy procesu.
Ewidentną zaletą tego podejścia jest zgodność z mechanika kwantową na poziomie lokalnych eksperymentów. Jest to nijakie rozszerzenie i enkapsulacja idei POVM i reguły Borna. Podejście te porzuca założenie globalnej struktury czasoprzestrzeni. W celu zachowania zgodności z mechaniką kwantową na poziomie lokalnym opiera się na następujących założeniu, że operacje wykonywane przez poszczególną stronę są opisywane przez mechanikę kwantową w standardowym przyczynowym sformułowaniu, które można opisywać przy pomocy zbioru \textit{kwantowych instrumentów} \cite{quantum_instrument} z wejściową przestrzenią Hilberta $\mathcal{H}^{A_1}$ i przestrzenią wyjściową  $\mathcal{H}^{A_2}$. Najogólniej można je realizować przy pomocy zadziałania unitarną transformacją na system wejściowy i ancillę, następnie wykonanie rzutującego pomiaru na części systemu pozostawiając pozostała część systemu jako wyjście. Alicja wykorzystując dany instrument otrzymuje jeden z możliwych wyników $x_i$, który indukuję transformację $\mathcal{M}^A_j$ z wejścia na wyjście. Transformacja ta odpowiada zupełnie dodatniej (CP) niezwiększającej śladu mapie 
\[\mathcal{M}^A_j : \mathcal{L}(\mathcal{H}^{A_1}) \mapsto \mathcal{L}(\mathcal{H}^{A_2})\] gdzie $\LHx{X}$ jest przestrzenią macierzy na $\Hx{X}$, której wymiar to $d_X$. Jej działanie na macierz gęstości $\rho$ opisuje następująca formuła
\[
\MAi{\rho} = \sum^m_j E_{ij} ^\dag \rho E_{ij}
\]
\end{multicols*}
\newpage
\bibliographystyle{plain}
\bibliography{bibliografia}

\end{document}
